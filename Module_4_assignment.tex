% Options for packages loaded elsewhere
\PassOptionsToPackage{unicode}{hyperref}
\PassOptionsToPackage{hyphens}{url}
%
\documentclass[
]{article}
\usepackage{amsmath,amssymb}
\usepackage{lmodern}
\usepackage{ifxetex,ifluatex}
\ifnum 0\ifxetex 1\fi\ifluatex 1\fi=0 % if pdftex
  \usepackage[T1]{fontenc}
  \usepackage[utf8]{inputenc}
  \usepackage{textcomp} % provide euro and other symbols
\else % if luatex or xetex
  \usepackage{unicode-math}
  \defaultfontfeatures{Scale=MatchLowercase}
  \defaultfontfeatures[\rmfamily]{Ligatures=TeX,Scale=1}
\fi
% Use upquote if available, for straight quotes in verbatim environments
\IfFileExists{upquote.sty}{\usepackage{upquote}}{}
\IfFileExists{microtype.sty}{% use microtype if available
  \usepackage[]{microtype}
  \UseMicrotypeSet[protrusion]{basicmath} % disable protrusion for tt fonts
}{}
\makeatletter
\@ifundefined{KOMAClassName}{% if non-KOMA class
  \IfFileExists{parskip.sty}{%
    \usepackage{parskip}
  }{% else
    \setlength{\parindent}{0pt}
    \setlength{\parskip}{6pt plus 2pt minus 1pt}}
}{% if KOMA class
  \KOMAoptions{parskip=half}}
\makeatother
\usepackage{xcolor}
\IfFileExists{xurl.sty}{\usepackage{xurl}}{} % add URL line breaks if available
\IfFileExists{bookmark.sty}{\usepackage{bookmark}}{\usepackage{hyperref}}
\hypersetup{
  pdftitle={Module 4 - Instructions},
  pdfauthor={Victoria Gruner},
  hidelinks,
  pdfcreator={LaTeX via pandoc}}
\urlstyle{same} % disable monospaced font for URLs
\usepackage[margin=1in]{geometry}
\usepackage{graphicx}
\makeatletter
\def\maxwidth{\ifdim\Gin@nat@width>\linewidth\linewidth\else\Gin@nat@width\fi}
\def\maxheight{\ifdim\Gin@nat@height>\textheight\textheight\else\Gin@nat@height\fi}
\makeatother
% Scale images if necessary, so that they will not overflow the page
% margins by default, and it is still possible to overwrite the defaults
% using explicit options in \includegraphics[width, height, ...]{}
\setkeys{Gin}{width=\maxwidth,height=\maxheight,keepaspectratio}
% Set default figure placement to htbp
\makeatletter
\def\fps@figure{htbp}
\makeatother
\setlength{\emergencystretch}{3em} % prevent overfull lines
\providecommand{\tightlist}{%
  \setlength{\itemsep}{0pt}\setlength{\parskip}{0pt}}
\setcounter{secnumdepth}{-\maxdimen} % remove section numbering
\ifluatex
  \usepackage{selnolig}  % disable illegal ligatures
\fi

\title{Module 4 - Instructions}
\author{Victoria Gruner}
\date{14/04/2021}

\begin{document}
\maketitle

\#In the next assignment we want to replicate some plots from the paper
``Female Socialization: How Daughters Affect Their Legislator Fathers'
Voting on Women's Issues'' (Washington, 2008). The paper explores
whether having a daughter makes politicians more sensitive to women's
rights issues and how this is reflected in their voting behavior. The
main identifying assumption is that after controlling for the number of
children, the gender composition is random. This might be violated if
families that have a preference for girls keep having children until
they have a girl. In this assignment we will prepare a dataset that
allows us to test whether families engage in such a ``female child
stopping rule''.

\hypertarget{setup}{%
\section{Setup}\label{setup}}

\#* Load the libraries ``Rio'' and ``tidyverse'' \#* Change the path of
the working directory to your working directory.

\#```\{r, message=FALSE\}

library(rio) library(tidyverse) library(magrittr)

setwd(``C:/Users/Victoria/Desktop/INTRO R/Module\_4'') \#```

\#* import the data sets \emph{basic.dta} and \emph{genold108.dta} \#*
create a subset of the 108th congress from the \emph{basic} dataset \#*
join this subset with the \emph{genold} dataset

basic.data \textless- import(``basic.dta'') genold.data \textless-
import(``genold108.dta'')

basic108.data \textless- filter(basic.data, congress==108)

combined.data \textless- left\_join(genold.data, basic108.data)

\hypertarget{data-preparation}{%
\section{Data preparation}\label{data-preparation}}

\begin{itemize}
\tightlist
\item
  check table 1 in the appendix of the paper and decide which variables
  are necessary for the analysis (check the footnote for control
  variables)
\item
  drop all other variables. -----\textgreater\#* Recode \emph{genold}
  such that gender is a factor variable and missing values are coded as
  NAs. ------\textgreater\#* Recode \emph{party} as a factor with 3
  levels (D, R, I)
\item
  Recode \emph{rgroup} and \emph{region} as factors.
\item
  generate variables for age squared and service length squared
\item
  create an additional variable of the number of children as factor
  variable
\end{itemize}

congress.data \textless- combined.data \%\textgreater\% select(name,
ngirls, anygirls, genold, totchi, party, rgroup,region, age, female,
white, srvlng )

congress.data\$genold \%\textless\textgreater\% na\_if("")
\%\textgreater\% as.factor()

\#alternative approach:
\#congress.data\(genold <- as.factor(congress.data\)genold)
\#congress.data\(genold[congress.data\)genold != `B'\&
congress.data\$genold != `G'{]} \textless- NA

congress.data\(party <- as.factor(congress.data\)party)

congress.data\(party <- recode(congress.data\)party, ``1''=``D'',
``2''=``R'', ``3''=``I'')

congress.data\(region <- as.factor(congress.data\)region)
congress.data\(rgroup <- as.factor(congress.data\)rgroup)

congress.data \textless- congress.data \%\textgreater\% mutate(age\_sq =
age\^{}2) congress.data \textless- congress.data \%\textgreater\%
mutate(srvlng\_sq = srvlng\^{}2) congress.data \textless- congress.data
\%\textgreater\% mutate(totchi\_factor = as.factor(totchi))

\hypertarget{replicationg-table-1-from-the-appendix}{%
\section{Replicationg Table 1 from the
Appendix}\label{replicationg-table-1-from-the-appendix}}

\#We haven't covered regressions in R yet. Use the function \emph{lm()}.
The function takes the regression model (formula) and the data as an
input. The model is written as \(y \sim x\), where \(x\) stands for any
linear combination of regressors (e.g.~\(y \sim x_1 + x_2 + female\)).
Use the help file to understand the function.

\#* Run the regression
\(total.children = \beta_0 + \beta_1 gender.oldest + \gamma'X\) where
\(\gamma\) stands for a vector of coefficients and \(X\) is a matrix
that contains all columns that are control
variables.\footnote{This is just a short notation instead of writing the full model with all control variables $totchi = \beta_0 + \beta_1 genold + \gamma_1 age + \gamma_2 age^2 + \gamma_3 Democrat + ... + \epsilon$ which quickly gets out of hand for large models.}

\#total children lm.reg\_tot \textless- lm(totchi \textasciitilde{}
party + region + rgroup+age+age\_sq+female+white+srvlng+srvlng\_sq +
genold, congress.data)

sd\_1 \textless-
summary(lm.reg\_tot)\(coefficients[22,2] beta_1 <- summary(lm.reg_tot)\)coefficients{[}22,1{]}

\#number of girls lm.reg\_g \textless- lm(ngirls \textasciitilde{} party
+ region + rgroup+age+age\_sq+female+white+srvlng+srvlng\_sq + genold,
congress.data)

sd\_1\_g \textless-
summary(lm.reg\_g)\(coefficients[22,2] beta_1_g <- summary(lm.reg_g)\)coefficients{[}22,1{]}

\#* Run the same regression separately for Democrats and Republicans
(assign the independent to one of the parties). Save the coefficient and
standard error of \emph{genold}

congress.rep \textless- filter(congress.data, party==`R')

lm.reg\_rep \textless- lm(totchi \textasciitilde{} genold + region
+rgroup+age+age\_sq+female+white+srvlng+srvlng\_sq, congress.rep)

sd\_1\_rep \textless-
summary(lm.reg\_rep)\(coefficients[2,2] beta_1_rep <- summary(lm.reg_rep)\)coefficients{[}2,1{]}

\#sd\_rep \textless- summary(lm.reg\_rep)\$coefficients{[}2,2{]}

lm.reg\_rep\_g \textless- lm(ngirls \textasciitilde{} genold + region
+rgroup+age+age\_sq+female+white+srvlng+srvlng\_sq , congress.rep)
sd\_1\_rep\_g \textless-
summary(lm.reg\_rep\_g)\(coefficients[2,2] beta_1_rep_g <- summary(lm.reg_rep_g)\)coefficients{[}2,1{]}

congress.dem \textless- filter(congress.data, party==`D' \textbar{}
party==`I')

lm.reg\_dem \textless- lm(totchi \textasciitilde{} genold + region
+rgroup+age+age\_sq+female+white+srvlng+srvlng\_sq , congress.dem)
sd\_1\_dem \textless-
summary(lm.reg\_dem)\(coefficients[2,2] beta_1_dem <- summary(lm.reg_dem)\)coefficients{[}2,1{]}

lm.reg\_dem\_g \textless- lm(ngirls \textasciitilde{} genold + region
+rgroup+age+age\_sq+female+white+srvlng+srvlng\_sq , congress.dem)
sd\_1\_dem\_g \textless-
summary(lm.reg\_dem\_g)\(coefficients[2,2] beta_1_dem_g <- summary(lm.reg_dem_g)\)coefficients{[}2,1{]}

\#* Collect all the \emph{genold} coefficients from the six regressions,
including their standard errors and arrange them in a table as in the
paper. \#* print the table

beta\_vector \textless- matrix(c(beta\_1, beta\_1\_g, beta\_1\_rep,
beta\_1\_rep\_g, beta\_1\_dem, beta\_1\_dem\_g, sd\_1, sd\_1\_g,
sd\_1\_rep, sd\_1\_rep\_g, sd\_1\_dem,sd\_1\_dem\_g), nrow=2, ncol= 6)
colnames(beta\_vector) \textless- c(`Total Children', `Girls',`Total
Children', `Girls',`Total Children', `Girls') rownames(beta\_vector)
\textless- c(`Beta', `Standard Error') beta\_vector \textless-
as.table(beta\_vector) beta\_vector

export(beta\_vector,``Congress.xls'')

\end{document}
